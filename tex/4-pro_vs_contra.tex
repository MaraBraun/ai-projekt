\section{Gegenüberstellung der Frameworks}
Dieses Kapitel soll die wesentlichen Anforderungen an Web Application Frameworks ausführen und eine Begründung enthalten, weshalb diese Anforderungen relavant sind. Es wird benötigt, um in den folgenden Kapiteln die Vor- und Nachteile der beiden Frameworks vergleichbar darstellen zu können.
\subsection{Anforderungen an Web Application Frameworks}
\subsubsection{Lizenz und Lizenzkosten}
Da es sowohl Open Source als auch lizenzierte Frameworks zur Entwicklung von Webanwendungen gibt, können sich die eventuell vorhandenen Lizenzkosten auf die Gesamtkosten des Projektes auswirken. Zudem müssen anfallende Lizenzkosten zum Betreiben der entwickelten Software bei der Auswahl eines Frameworks mit betrachtet werden.
Ein weiterer zu beachtender Punkt bezüglich der Lizenzen ist die GNU General Public License (GNU GPL), denn sobald Teile des Quellcodes einer Software die unter diese Lizenz fällt in der eigenen Anwendung verwendet werden kann die Notwendigkeit bestehen, dass die eigene Anwendung ebenfalls unter diese Lizenz fallen muss. Daher ist insbesondere bei der Verwendung von GNU GPL Erweiterungen zu prüfen ob die Anwendung bei Verwendung ebenfalls unter die GNU GPL fällt.\citep[S.214]{EFCMW2013} 
\subsubsection{Entwicklungsgeschwindigkeit}
Die mögliche Geschwindigkeit mit der mit einem Framework entwickelt werden kann ist sehr entscheidend bei der Wahl des Frameworks, da die Entwicklungszeit und damit auch die Kosten für den Kunden möglichst gering gehalten werden sollen. Aus diesem Grund sollte auf diesen Aspekt geachtet werden. In die tatsächliche Entwicklungsdauer fließt jedoch auch noch die Lernkurve mit ein.\citep[S.214]{EFCMW2013}
\subsubsection{Lernkurve}
Da Web Application Frameworks nicht alle gleich aufgebaut oder ähnlich umfangreich sind, kann auch die Einarbeitungszeit für das entsprechende Framework unterschiedlich lang ausfallen. Da dies die Entwicklungszeit für die entsprechende Software verlängern oder verkürzen kann, gilt es die Lernkurve bei Auswahl des Frameworks zu beachten.\citep[S.214]{EFCMW2013}
\subsubsection{Community}
Für die Entwicklung mit einem entsprechenden Framework kann die vorhandene Community ein essentieller Bestandteil sein, denn immer wieder können bei der Entwicklung Fehler oder Probleme auftreten, deren Lösung viel Zeit in Anspruch nehmen kann. In einer großen Community kann das Problem aber bereits von einem anderen Entwickler gelöst und festgehalten worden sein oder aber es wird auf Anfrage von jemandem gelöst, was einiges an Zeit einsparen kann.
\subsubsection{Testbarkeit}
Die Testbarkeit von Software ist allgemein sehr wichtig, da Software zum einen komplex ist und sich zum anderen häufig ändern kann. Auf Grund dieser beiden Aspekte, ist es möglich, dass durch Änderungen Fehler entstehen können die in der komplexen Software vielleicht zunächst nicht auffallen. Eine gute Testabdeckung kann verhindern, dass solche Fehler übersehen werden.
\subsubsection{Umfang und Qualität der Dokumentation}
Eine umfangreiche und verständliche Dokumentation ist sehr wichtig um mit dem Framework arbeiten zu können. Eine schlechte Dokumentation führt schnell zu Verwirrung und Frustration und verlangsamt damit auch die Entwicklungsgeschwindigkeit. Eine umfangreiche und mit Beispielen versehene Dokumentation ist daher wichtig für die Auswahl des Frameworks.\citep[S.214]{EFCMW2013}
\subsubsection{Modularität}
Da es sehr viele verschiedene Verwendungsmöglichkeiten für Web Anwendungen mit sehr unterschiedlichen Funktionsanforderungen gibt, ist es häufig sinnvoll nicht alle möglichen Funktionalitäten im Framework zu integrieren, sondern diese durch Erweiterungen (im nächsten Punkte näher erläutert) verfügbar zu machen. Ein leichtgewichtigeres Framework ist für kleine Webanwendungen daher besser geeignet, da weniger nicht benötigte Funktionen enthalten sind und die Komplexität des Frameworks nicht unnötig vergrößern.\citep[S.214]{EFCMW2013}
\subsubsection{Erweiterbarkeit}
Bei der Entwicklung einer Web Anwendung ist es gut möglich, dass Funktionen verwendet werden sollen, die für sehr viele Web Anwendungen die selben sind (z.B. die Authentifizierung). Daher ist die Erweiterbarkeit des Frameworks und die Möglichkeit der Verwendung von Plugins für Web Application Frameworks ein wesentlicher Aspekt.\citep[S.214]{EFCMW2013}
\subsubsection{Versionierbarkeit}
Da Software und damit auch Webanwendungen häufig im Team entwickelt werden, muss eine Möglichkeit zur Versionierung der zu entwickelnden Anwendung gegeben sein. Hierbei ist es wünschenswert, dass beim Zusammenführen der verschiedenen Entwicklungsstände möglichst wenig Konflikte auftreten. Weiterhin sollte gute Versionierung Änderungen im Nachhinein nachvollziehbar machen.
\subsubsection{Langlebigkeit}
Die Langlebigkeit eines Frameworks ist insbesondere im Bezug auf den zugehörigen Support und Sicherheitsupdates relevant, da diese mit dem Aussterben des Frameworks ebenfalls wegfallen. Es ist also für eine Anwendung die für eine lange Dauer zum Einsatz kommen soll zu beachten, dass auch das Framework für diese Dauer warscheinlich weiterhin unterstützt wird.\citep[S.214]{EFCMW2013}
\subsubsection{Performanz}
Die Performanz von Web Application Frameworks und den daraus resultierenden Anwendungen ist ebenfalls ein wesentlicher Punkt bei der Auswahl eines Frameworks, jedoch wird dieser hier nicht genauer betrachtet, da es schwer ist sinnvoll vergleichbare Werte für die Performanz zweier Frameworks zu finden.
\subsection{Vor- und Nachteile von ADF}
Da ADF eine proprietäre kommerzielle Software ist fallen bei Einsatz der, mit diesem Framework entwickelten, Software Lizenzkosten an. Jedoch hat die Verwendung eines proprietären Frameworks den Vorteil, das der Quellcode in keinem Fall offen gelegt werden muss, wie es bei der GNU GPL\citep{GNU2015} der Fall ist, was für einige Kunden essentiell ist. Der Aufbau des Frameworks ADF lässt jedoch nicht als modular bezeichnen, da ADF ein kaum erweiterbares Framework ist und entsprechend den gesamten Umfang der möglichen Funktionen schon zu Beginn enthält. Dieser große Umfang des Frameworks wirkt sich wiederum auf die Lernkurve des Frameworks aus und bewirkt, dass das Framework insbesondere für Entwickler mit wenig Erfahrung im Java Enterprise Umfeld schwerer zu erlernen ist\citep[S.24]{AUW2009}. Einen Überblick über das Framework ADF zu bekommen kann daher einiges an Zeit erfordern und wirkt sich damit negativ auf die Entwicklungsdauer aus. Einem Entwickler, der bereits Erfahrung mit ADF gesammelt hat und mit ADF vertraut ist, ist das sogenannte "`Rapid Application Development"' möglich, das ADF durch seinen baukastenartigen Aufbau und die vielen Wizards ermöglicht \citep[S.3]{ARIA2015}. Dieses schnelle Entwicklungstempo wird zudem durch die umfangreiche Dokumentation und den Support von Oracle selbst durch zahlreiche Tutorials für die unterschiedlichsten Problemszenarien unterstützt. Die zu ADF gehörige Community ist weltweit deutlich kleiner als die zum Framework Grails gehörige. Dies lässt sich anhand der folgenden Abbildung erkennen. 
\begin{figure}[h]
\centering
\includegraphics[width=\textwidth]{img/interesse_zeitl.png}
\caption{Zeitverlauf des Interesses an ADF und Grails weltweit \citep{GT2015} }
\end{figure}
Die Grafik zeigt den zeitlichen Verlauf des Interesses an den beiden Frameworks ADF (in rot) und Grails (in blau) weltweit anhand der Häufigkeit\footnote{für genauere Erläuterung zur Wertefindung im Diagramm: http://goo.gl/Ma9kWJ} von Suchanfragen über Google. Das Suchinteresse wird hierbei relativ zum Höchstwert dargestellt. Diese Grafik lässt den Schluss zu, dass ADF zwar eine kleinere Community hat aber sich durchaus als langlebig erwiesen hat, da das Interesse an ADF mehr oder weniger konstant bleibt. Die Langlebigkeit des Frameworks, die auch in Zukunft zu erwarten ist bietet den Vorteil, dass auch der Support weiterhin besteht und damit auch spätere Wartungsaufgaben erfüllbar sind \citep[S.214]{EFCMW2013}. Auch die im Anforderungskatalog aufgeführten Punkte Versionierbarkeit und Testbarkeit erfüllt ADF, da es sich zum einen mit den bekannten Versionierungstools GIT und SVN versionieren lässt und damit auch die Arbeit in größeren Teams zulässt, zum anderen aber auch beispielsweise Unit Tests möglich sind um z.B. selbst erstellte Java Klassen und Methoden testen zu können. Bei der Versionierbarkeit gilt es jedoch zu beachten, dass nicht jeder Schritt für andere Entwickler Nachvollziehbar bleibt, da ADF viele Wizards zur Erstellung von Funktionen, Klassen und Dateien bietet. Diese Wizards sind aus der Sicht von GIT und SVN die rein textbasiert arbeiten nicht immer nachvollziehbar.

\subsection{Vor- und Nachteile von Grails}
Grails ist ein modulares Open Source Framework, dass unter die Apache License in der Version 2.0 fällt. Diese Lizenz hat zunächst keine Auswirkungen auf die zu entwickelnde Anwendung, da im allgemeinen diese Lizenz nur eine Auswirkung hätte, wenn das Framework Teile des eigenen Source Codes in das Programm kopieren würde \citep{AL2015}. Dies kann aber bei Grails höchstens bei Plugins (den zahlreichen Erweiterungsmöglichkeiten von Grails) vorkommen, die wiederum alle ihre eigenen Lizenzen haben. Dies hat wiederum den Nachteil, dass man für jede der vielen möglichen Erweiterungen des Frameworks, die man verwenden möchte zunächst überprüfen muss ob die Lizenz Auswirkungen auf die zu entwickelnde Software hat. Diese Überprüfungen können wenn sie notwendig sind, negative Auswirkungen auf die Entwicklungsgeschwindigkeit haben. Ansonsten sind Entwicklungsgeschwindigkeit und Lernkurve für Entwickler mit Java Vorkenntnissen sehr gut, da diese Grails schnell lernen können und entsprechend schnell entwickeln können. Grails bietet allerdings weniger Wizards und ist nicht wie ADF nach dem Baukasten Prinzip für "`Rapid Application Development"' ausgelegt. Es muss also der meiste Code selbst geschrieben werden. Durch sehr viel selbst geschriebenen Code wird sowohl eine umfangreiche Testabdeckung, als auch eine gute Dokumentation und Commmunity benötigt. Diese Punkte sind gegeben, da Grails die Java basierte Programmmiersprache Groovy verwendet und damit zum einen die gleichen Testmöglichkeiten bestehen, zum anderen aber auch die Javadocs als Dokumentation gültig sind, welche sehr umfangreich sind. Zusätzlich zu der Java Dokumentation hat Grails aber auch eine eigene umfangreiche Dokumentation und eine große Community, die in entsprechenden Foren Fragen beantwortet und Beispiele für verschiedene Problemstellungen zur Verfügung stellt. Wie auch für ADF lässt sich die Größe der Community etwa am Zeitverlauf des Interesses an den beiden Frameworks (siehe Abbildung 5: Zeitverlauf des Interesses an ADF und Grails weltweit
) und an den Einträgen in dem wohl meist genutzten Portal "`Stackoverflow"'\citep{SOFG2015} ablesen. Was die Langlebigkeit von Grails betrifft so ist das Interesse an Grails weniger konstant, jedoch ist es hoch genug um davon ausgehen zu können, dass Grails noch einige Zeit bestehen wird. Ein Weiterer für Webapplication Frameworks relevanter Punkt ist die Versionierbarkeit. Das Versionieren der zu entwickelnden Software via z.B. SVN oder GIT ist problemlos möglich. Hierbei hat Grails jedoch den Vorteil, dass durch den vielen eigenen Code und weniger Wizards die ausgeführten Schritte später leichter nachzuvollziehen sind, was für Entwicklung in großen Teams ein großer Vorteil ist.