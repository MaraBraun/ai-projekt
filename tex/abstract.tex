\section*{Abstract}
Die OPITZ CONSULTING Deutschland GmbH ist seit einigen Jahren mit der Wartung einer Webanwendung zur Verwaltung von Bürgschaften betraut. Da diese Anwendung jedoch schon vor einigen Jahren entwickelt wurde, entspricht ihre Performanz nicht mehr den Anforderungen der heutigen Zeit. Daher soll die Anwendung neu entwickelt werden. Hierfür kommen zwei große Frameworks in Frage. Das erste Framework ist das Application Development Framework von Oracle, in dessen veralteter Version  die zugrunde liegende Anwendung ursprünglich erstellt wurde. Das zweite Framework ist das zur Zeit verbreitete Open Source Framework Grails.

Im folgenden Dokument sollen nun die beiden Frameworks miteinander verglichen werden. Dieser Vergleich soll die Vor und Nachteile der beiden Frameworks gegenüberstellen, um zu evaluieren, welches sich besser zur Neuentwicklung der Bürgschaftsverwaltung eignet.