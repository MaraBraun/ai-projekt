\section{Gegenüberstellung der Frameworks}
\subsection{Anforderungen an Web Application Frameworks}
Dieses Kapitel soll die wesentlichen Anforderungen an Web Application Frameworks zusammenfassen und eine Begründung enthalten, weshalb diese Anforderungen relavant sind. Es wird benötigt, um in den Folgenden Kapiteln die Vor und Nachteile der beiden Frameworks vergleichbar darstellen zu können.
\subsubsection*{Lizenz und Lizenzkosten}
Da es sowohl Open Source als auch lizensierte Frameworks zur Entwicklung von Webanwendungen gibt, können sich die eventuell vorhandenen Lizenzkosten auf die Gesamtkosten des Projektes auswirken. Zudem müssen anfallende Lizenzkosten zum Betreiben der entwickelten Software bei der Auswahl eines Frameworks mit betrachtet werden.
\subsubsection*{Lernkurve}
Da Web Application Frameworks nicht alle gleich aufgebaut sind oder ähnlich umfangreich sind, kann auch die Einarbeitungszeit für das entsprechende Framework unterschiedlich lang ausfallen. Da dies die Entwicklungszeit für die entsprechende Software verlängern oder verkürzen kann, gilt es die Lernkurve bei Auswahl des Frameworks zu beachten.
\subsubsection*{Community und Support}
Für die Entwicklung mit einem entsprechenden Framework kann die vorhandene Community und der Support ein essentieller Bestandteil sein, denn immer wieder können bei der Entwicklung Fehler oder Probleme auftreten, deren Lösung viel Zeit in Anspruch nehmen kann. In einer großen Community kann das Problem aber bereits von einem anderen Entwickler gelöst und festgehalten worden sein, was einiges an Zeit einsparen kann.
\subsubsection*{Mehrsprachigkeit}
In vielen Fällen ist es heutzutage wichtig, dass eine Anwendung in mehreren Sprachen verfügbar ist. Daher ist die Komplexität zur Umsetzung dieser Mehrsprachigkeit wesentlich für die Auswahl des Frameworks.
\subsubsection*{Testbarkeit}
Die Testbarkeit von Software ist allgemein sehr wichtig, da sie zum einen komplex ist und sich zum anderen häufig ändern kann. Auf Grund dieser beiden Aspekte, ist es möglich, dass durch Änderungen Fehler entstehen können die in der komplexen Software vielleicht zunächst nicht auffallen. Eine gute Testabdeckung kann verhindern, dass solche Fehler übersehen werden.
\subsubsection*{Umfang und Qualität der Dokumentation}
Eine Umfangreiche und verständliche Dokumentation ist sehr wichtig um mit dem Framework arbeiten zu können. Eine schlechte Dokumentation führt schnell zu Verwirrung und Frustration und verlangsamt damit auch die Entwicklungsgeschwindigkeit. Eine Umfangreiche und mit Beispielen versehene Dokumentation ist daher wichtig für die Auswahl des Frameworks.
\subsubsection*{Erweiterbarkeit}
Bei der Entwicklung einer Web Anwendung ist es gut Möglich, dass Funktionen verwendet werden sollen, die für sehr viele Web Anwendungen die selbe ist (z.B. das Login). Daher ist die Erweiterbarkeit des Frameworks und die Möglichkeit der Verwendung von Plugins für Web Application Frameworks ein wesentlicher Aspekt.
\subsubsection*{Versionierbarkeit}
Da Software und damit auch Webanwendungen häufig im Team entwickelt werden, muss eine Möglichkeit zur Versionierung der zu entwickelnden Anwendung gegeben sein. Hierbei ist es wünschenswert, dass beim Zusammenführen der verschiedenen Entwicklungsstände möglichst wenig Konflikte auftreten.
\subsubsection*{Performanz}
Die Performanz von Web Application Frameworks und den daraus resultierenden Anwendungen ist ebenfalls ein wesentlicher Punkt bei der Auswahl eines Frameworks, jedoch wird dieser hier nicht genauer betrachtet, da es schwer ist sinnvoll vergleichbare Werte für die Performanz zweier Frameworks zu finden.
\subsection{Vor- und Nachteile von ADF}
Was zeichnet ADF aus? Was kann ADF besser als andere Frameworks seiner Art?
Welche Nachteile bringt ADF, was macht es schlechter als vergleichbare Frameworks?
hat lizenzkosten
\subsection{Vor- und Nachteile von Grails}
Was zeichnet Grails aus? Was kann Grails besser als andere Frameworks seiner Art?
Welche Nachteile bringt Grails, was macht es schlechter als vergleichbare Frameworks?
ist opensource und hat keine lizenzkosten
mit Java Kenntnissen schnell zu lernen

Hibernate

