\section{Einleitung}
Im Zeitalter des Internets sind Web Anwendungen und mit ihnen Web Application Frameworks immer beliebter geworden, weshalb es in der heutigen Zeit eine Vielzahl von Web Application Frameworks gibt. Dies erschwert die Auswahl eines geeigneten Frameworks für das eigene Projekt, was auch allgemein ein relavantes Thema in Softwarehäusern, wie z.B. der OPITZ CONSULTING Deutschland GmbH ist. Die OPITZ CONSULTING Deutschland GmbH ist seit einigen Jahren mit der Wartung der unterschiedlichsten Anwendungen betraut und erhält ebenfalls regelmäßig Aufträge zur Entwicklung neuer Software. Zu den zu wartenden Anwendungen gehört unter anderem die Web Anwendung "`AvaleNet"' eines Bankhauses, die in einer veralteten Version des Frameworks Application Development Framework (ADF) entwickelt wurde und nun neu entwickelt werden soll, da die Performanz der Anwendungen nicht mehr den Anforderungen der heutigen Zeit genügt. Hierzu muss eine Entscheidung getroffen werden, welches Framework zur Migration der Anwendung verwendet werden soll. Die Entscheidungsfindung, welches der beiden zur Auswahl gestellten Frameworks verwendet werden soll, ist Bestandteil des folgenden Dokumentes. Hierfür werden zunächst die fachlichen Details der beiden Frameworks erläutert (Kapitel \ref{cha:kapitel-2}), um im darauf folgenden Kapitel (Kapitel \ref{cha:kapitel-3}) die Vor- und Nachteile der Beiden besser nachvollziehen zu können. Diese Vor- und Nachteile, die mit Hilfe einer zur Vergleichbarkeit entwickelten Metrik ermittelt wurden (Kapitel \ref{sec:anforderungskatalog}), werden dann in Bezug zu der zu entwickelnden Anwendung "`AvaleNet"' gesetzt und gegenübergestellt (Kapitel \ref{cha:kapitel-4}). Zuletzt wird eine Gewichtung der einzelnen Kriterien vorgenommen (Kapitel \ref{sec:gewichtung}), anhand derer letztendlich entschieden wird, welches Framework sich am besten zur Migration der Anwendung eignet (Kapitel \ref{cha:fazit}).