\section{Fazit}
\label{cha:fazit}
Das Ziel dieses Dokumentes ist es ein geeignetes Web Application Framework zu ermitteln, um die Web Anwendung "`AvaleNet"' eines Bankhauses in eine neue Version zu migrieren. Hierbei ergab sich, dass sich im Gegensatz zu dem proprietären Application Development Framework (ADF) von Oracle das Open Source Web Application Framework Grails am besten eignet, um "`AvaleNet"' in eine neue Version zu migrieren. Dies ist damit zu Begründen, dass unter Anderem für eine Grails Anwendung im Gegensatz zu einer ADF Anwendung keine zusätzlichen Lizenzkosten anfallen. Des weiteren ist Grails ein modulares Framework, das sich nahezu beliebig erweitern lässt. Dies bietet gegenüber ADF den Vorteil, dass sich der Entwickler nur mit den benötigten Funktionen des Frameworks auseinandersetzen muss und es damit deutlich schneller erlernen kann.
Bezogen auf die Ausgangssituation, dass die neue Anwendung von einem noch unerfahrenen Entwickler mit Java Vorkenntnissen entwickelt werden soll, wirkt sich die Verwendung von Grails zudem positiv auf die Entwicklungsgeschwindigkeit und -dauer aus.

Allgemein kann man jedoch nicht sagen welches Framework besser ist als das andere, da dies immer davon abhängt, wozu es verwendet werden soll.


Zu diesem Zweck sind zunächst die den beiden, von der OPITZ CONSULTING Deutschland GmbH vorgegebenen, Frameworks zu Grunde liegende MVC-Architektur (Kapitel \ref{sec:mvc}) erläutert um in den nachfolgenden Kapiteln deren spezifische Architektur besser nachvollziehen zu können (Kapitel \ref{sec:adf} und \ref{sec:grails}). Die Architektur und die Eigenschaften der beiden Frameworks Grails und ADF sind wiederum von Interesse, um deren Vor- und Nachteile anhand der zuvor erstellten Metrik erläutern zu können (Kapitel \ref{cha:kapitel-3}). Diese werden dann in Kapitel \ref{cha:kapitel-4} gegenübergestellt und in Bezug zu der zu entwickelnden Anwendung "`AvaleNet"' gesetzt.