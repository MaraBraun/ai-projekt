\section{Projektspezifische Gegenüberstellung}
\subsection{Avalenet}
Dieses Kapitel soll die zu erstellende Anwendung Avalenet grundlegend beschreiben um einen Einblick zu erhalten, welche Aspekte der beiden Frameworks ADF und Grails für die Anwendung relevant sein können.
\subsubsection{Projekthintergrund}
Die zu migrierende Anwendung „AvaleNet“ eines Bankhauses zur Verwaltung von Kunden-, Konto- und Aval-Daten ist vor fünf Jahren in einer heute veralteten Version (Version 10) des Frameworks ADF entwickelt worden. Da dies eine heute schlechte Performanz der Anwendung und eine erschwerte Wartbarkeit zur Folge hat, soll die Anwendung nun migriert werden.
\subsubsection{Projektanforderungen}
Die Anwendung „AvaleNet“ ermöglicht es einem Mitarbeiter des Bankhauses zum einen bereits bestehende Daten von Kunden, Konten oder Avalen bearbeiten, zum anderen stehen ihm aber auch einige Export-Funktionen zur Verfügung. Er kann beispielsweise eine Liste aller, zu einem Kunden gehörigen Avale in ein PDF-Dokument exportieren, oder die Einzeldaten pro Aval auf diese Weise auflisten lassen.


\subsection{Vor- und Nachteile der Frameworks im Bezug auf Avalenet}
Welche Vorteile haben die beiden Frameworks im Bezug auf die zu entwickelnde Anwendung?
Welche Nachteile ergeben sich für Avalenet, wenn eines der Frameworks verwendet wird.
Bezüglich der zu entwickelnden Anwendung Avalenet hat ADF den Vorteilm dass es ein proprietäres Framework ist und damit keinesfalls der Quellcode der Anwendung offengelegt werden muss. Dies ist gerade bei einer Anwendung für ein Bankhaus relevant, die mit sensiblen Kundendaten agiert. Allerdings ist auch die Verwendung von Grails hier nicht kritisch, da kaum zusätzliche Plugins werden werden und diese (Namen + Lizenzen) keine kritischen Lizenzen enthalten, (prüfen -> Hibernate ??? -> Artikel gefunden) die den Entwickler zwingen die zu entwickelnde Software ebenfalls zu einer Open Source Software zu machen. Der Grund für die geringe Notwendigkeit von Plugins ist, dass Avalenet eine kleine Anwendung ist, die keine umfangreiche Funktionalität erfordert/enthält. Dies hat ebenfalls zur Folge, dass die Entwicklung der neuen Anwendung kein großes Team erfordert, sondern von einer Person mit geringem Aufwand (< 1/2 Jahr) umgesetzt werden kann. Da die entwickelnde Person zudem hauptsächlich Kenntnisse um Java Umfeld hat und keine Erfahrung mit ADF spielt die Lernkurve hier für die gesamte Entwicklungsdauer inklusive Einarbeitungszeit eine große Rolle. Hier hat/bring Grails den Vorteil, dass es mit Java Grundkenntnissen schneller zu erlernen ist als ADF ohne Vorkenntnisse. Ein weiterer Nachteil von ADF bezüglich Avalenet ist der Überfluss an Funktion, die das Framework bereitstellt, welche aber für eine solche kleine Anwendung nicht benötigt werden und es dem Entwickler erschweren das Framework schneller zu erlernen. Unabhängig von dem im vorigen Kapitel aufgestellten Anforderungskatalog hat ADF jedoch den Vorteil, dass die zu migrierende Anwendung bereits in ADF geschrieben wurde und ein erfahrener Entwickler die Anwendung daher mithilfe der alten Anwendung deutlich schneller in der neuen Version von ADF (12.0.3) nachbauen könnte. Allerdings sind in dem mit der Wartung und Migration betrauten Unternehmen (OPITZ Consulting) nur wenige Mitarbeiter mit umfangreichen ADF Kenntnissen dafür aber einige mit guten Java und Grailskenntnissen, weshalb Grails für diese Situation von Vorteil wäre.

Verwendungszweck
Skill im Untenehmen
Entwicklungsgeschwindigkeit

aktuell ADF kein Umgewöhnen + lizenzen vorhanden

\subsection{Aufwiegen der Vor- und Nachteile}
Dieses Kapitel fasst die Vor und Nachteile der beiden Frameworks zusammen und stellt sie vergleichend gegenüber, um letztendlich zu entscheiden, welches der beiden Frameworks sich besser zur Entwicklung der Anwendung Avalenet eignet.
Sehr wichtige Punkte bei der Auswahl des Frameworks zur Migration der Anwendung Avalenet sind eine kurze Entwicklungsdauer inklusive Einarbeitungszeit. Hierbei ist es insbesondere wichtig, dass die Einarbeitungszeit die Entwicklungszeit nicht übersteigt. Ebenfalls wichtig ist zudem die
