\section{Projektspezifische Gegenüberstellung}
\subsection{Avalenet}
Dieses Kapitel soll die zu erstellende Anwendung Avalenet grundlegend beschreiben um einen Einblick zu erhalten, welche Aspekte der beiden Frameworks ADF und Grails für die Anwendung relevant sein können.
\subsubsection{Projekthintergrund}
Die zu migrierende Anwendung „AvaleNet“ eines Bankhauses zur Verwaltung von Kunden-, Konto- und Aval-Daten ist vor fünf Jahren in einer heute veralteten Version (Version 10) des Frameworks ADF entwickelt worden. Da dies eine heute schlechte Performanz der Anwendung und eine erschwerte Wartbarkeit zur Folge hat, soll die Anwendung nun migriert werden.
\subsubsection{Projektanforderungen}
Die Anwendung „AvaleNet“ ermöglicht es einem Mitarbeiter des Bankhauses zum einen bereits bestehende Daten von Kunden, Konten oder Avalen bearbeiten, zum anderen stehen ihm aber auch einige Export-Funktionen zur Verfügung. Er kann beispielsweise eine Liste aller, zu einem Kunden gehörigen Avale in ein PDF-Dokument exportieren, oder die Einzeldaten pro Aval auf diese Weise auflisten lassen.


\subsection{Vor- und Nachteile der Frameworks im Bezug auf Avalenet}
Welche Vorteile haben die beiden Frameworks im Bezug auf die zu entwickelnde Anwendung?
Welche Nachteile ergeben sich für Avalenet, wenn eines der Frameworks verwendet wird.
\subsection{Aufwiegen der Vor- und Nachteile}
Dieses Kapitel fasst die Vor und Nachteile der beiden Frameworks zusammen und stellt sie vergleichend gegenüber, um letztendlich zu entscheiden, welches der beiden Frameworks sich besser zur Entwicklung der Anwendung Avalenet eignet.